\chapter{Introdução}\label{capitulo1}

"O empreendedor é uma pessoa que enxerga uma \textbf{oportunidade} aliada a um \textbf{sonho} ou uma ideia, tem \textbf{coragem} para colocá-la em prática e criar um \textbf{diferencial} relacionado ao novo negócio, um projeto social ou mesmo uma inovação dentro do ambiente de trabalho" (Santos e Souza2013 Falta achar referência). Quando posta em prática, a oportunidade se torna um produto (tangível ou intangível) que pode vir a trazer importantes benefícios à sociedade, como foram os grandes inventos: avião e computador.

 O termo \textit{empreendedor} data de um período remoto. Os primeiros empreendedores surgiram no Século XIII, período do explorador Marco Polo, atuando de forma ativa no comércio através da rota com o oriente. No Brasil, a existência de empreendedores se intensificou nos anos 90, e consolidou-se em 2000. Isso se deve a variações econômicas que incentivam aos de "mente inovadora” a criarem novas alternativas e soluções para o mercado. Atualmente existem empresas especializas em desenvolver esse espírito do empreendedor de forma técnica, tornando os empreendedores também bons administradores, por exemplo, o Sebrae e a Empretec. Eventos patrocinados ou incentivados por tais empresas costumam cultivar a criação de startups que são empresas com foco em funcionamento imediato e com forte potencial de crescimento. Muitas empresas hoje consolidadas e com alto capital surgiram como startups. Entre essas empresas estão a Google, Facebook, Uber e Airbnb.
    
As Startups são comumente associadas à tecnologia e fundadas por membros jovens. O IFNMG - \textit{Campus} Montes Claros oferece cursos técnicos de informática e química, duas áreas com amplo potencial para inovação. Diante desse fato, o presente trabalho pretende analisar o perfil empreendedor dos alunos de ambos os cursos citados, afim de compreender a visão dos alunos sobre o empreendedorismo e possivelmente desenvolver eventos, atividades, palestras e dinâmicas que possam os instigar e orientar para serem empreendedores de sucesso.